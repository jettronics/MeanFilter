% !TEX TS-program = pdflatex
% !TEX encoding = UTF-8 Unicode

% This is a simple template for a LaTeX document using the "article" class.
% See "book", "report", "letter" for other types of document.

% \documentclass[11pt]{article} % use larger type; default would be 10pt
\documentclass[
    10pt, % Schriftgröße
    DIV12,
    english, % für Umlaute, Silbentrennung etc.
    a5paper, % Papierformat
    twoside, % zweiseitiges Dokument
    titlepage, % es wird eine Titelseite verwendet
    parskip=half, % Abstand zwischen Absätzen (halbe Zeile)
    headings=small, % Größe der Überschriften verkleinern
    listof=totoc, % Verzeichnisse im Inhaltsverzeichnis aufführen
    bibliography=totoc, % Literaturverzeichnis im Inhaltsverzeichnis aufführen
    index=totoc, % Index im Inhaltsverzeichnis aufführen
    captions=tableheading, % Beschriftung von Tabellen unterhalb ausgeben
    final % Status des Dokuments (final/draft)
]{scrbook}

\usepackage[utf8]{inputenc} % set input encoding (not needed with XeLaTeX)
\usepackage[T1]{fontenc}
\usepackage{textcomp} % Euro-Zeichen etc.

\usepackage[
	automark, % Kapitelangaben in Kopfzeile automatisch erstellen
	headsepline, % Trennlinie unter Kopfzeile
	ilines % Trennlinie linksbündig ausrichten
]{scrlayer-scrpage}

% Anpassung an Landessprache ---------------------------------------------------
\usepackage[english]{babel}

%%% Examples of Article customizations
% These packages are optional, depending whether you want the features they provide.
% See the LaTeX Companion or other references for full information.

\usepackage{lmodern} % bessere Fonts
\usepackage{relsize} % Schriftgröße relativ festlegen

%%% PAGE DIMENSIONS
% \usepackage{geometry} % to change the page dimensions
\usepackage{setspace}
\usepackage[paperwidth=17cm,paperheight=24cm]{geometry}
% \geometry{a4paper} % or letterpaper (US) or a5paper or....
% \geometry{margin=2in} % for example, change the margins to 2 inches all round
% \geometry{landscape} % set up the page for landscape
%   read geometry.pdf for detailed page layout information
\geometry{left=20mm,right=20mm,top=15mm,bottom=20mm}

\usepackage{graphicx} % support the \includegraphics command and options

% \usepackage[parfill]{parskip} % Activate to begin paragraphs with an empty line rather than an indent

%%% PACKAGES
\usepackage{trfsigns}
\usepackage{booktabs} % for much better looking tables
\usepackage{array} % for better arrays (eg matrices) in maths
\usepackage{paralist} % very flexible & customisable lists (eg. enumerate/itemize, etc.)
\usepackage{verbatim} % adds environment for commenting out blocks of text & for better verbatim
\usepackage{subfig} % make it possible to include more than one captioned figure/table in a single float
\usepackage{float}
\usepackage{amsmath} %
\usepackage{amssymb}
\usepackage{tikz} % Draw sig­nal flow graphs
\usetikzlibrary{dsp,chains}
% These packages are all incorporated in the memoir class to one degree or another...

%%% HEADERS & FOOTERS
% \usepackage{fancyhdr} % This should be set AFTER setting up the page geometry
% \pagestyle{fancy} % options: empty , plain , fancy
% \renewcommand{\headrulewidth}{0pt} % customise the layout...
% \lhead{}\chead{}\rhead{}
% \lfoot{}\cfoot{\thepage}\rfoot{}

% Kopf- und Fußzeilen ----------------------------------------------------------
\pagestyle{scrheadings}
% Kopf- und Fußzeile auch auf Kapitelanfangsseiten
\renewcommand*{\chapterpagestyle}{scrheadings} 
% Schriftform der Kopfzeile
\renewcommand{\headfont}{\normalfont}

% Kopfzeile
\ihead{\textit{\headmark}}
\chead{}
\ohead{\pagemark}
\setlength{\headheight}{10mm} % Höhe der Kopfzeile
\setheadsepline[text]{0.4pt} % Trennlinie unter Kopfzeile

% Fußzeile
\ifoot{}
\cfoot{}
\ofoot{}

\setlength{\footskip}{20pt}

%%% SECTION TITLE APPEARANCE
%\usepackage{sectsty}
%\allsectionsfont{\sffamily\mdseries\upshape} % (See the fntguide.pdf for font help)
% (This matches ConTeXt defaults)

\RedeclareSectionCommand[
  beforeskip=-1sp,
  afterskip=2\baselineskip]{chapter}
\RedeclareSectionCommand[
  beforeskip=-\baselineskip,
  afterskip=.5\baselineskip]{section}
\RedeclareSectionCommand[
  beforeskip=-.75\baselineskip,
  afterskip=.5\baselineskip]{subsection}
\RedeclareSectionCommand[
  beforeskip=-.5\baselineskip,
  afterskip=.25\baselineskip]{subsubsection}
\RedeclareSectionCommand[
  beforeskip=.5\baselineskip,
  afterskip=-1em]{paragraph}
\RedeclareSectionCommand[
  beforeskip=-.5\baselineskip,
  afterskip=-1em]{subparagraph}

%%% ToC (table of contents) APPEARANCE
\usepackage[nottoc,notlof,notlot]{tocbibind} % Put the bibliography in the ToC
\usepackage[titles,subfigure]{tocloft} % Alter the style of the Table of Contents
\renewcommand{\cftsecfont}{\rmfamily\mdseries\upshape}
\renewcommand{\cftsecpagefont}{\rmfamily\mdseries\upshape} % No bold!

\DeclareMathAlphabet{\mathpzc}{OT1}{pzc}{m}{it}
\newcommand{\z}{\mathpzc{z}}

\usepackage{xcolor}
\definecolor{instruction}{gray}{0.80}

\newtheorem{beispiel}{Beispiel}[chapter]

% zum Einbinden von Programmcode -----------------------------------------------
\usepackage{listings}
\usepackage{xcolor} 
\definecolor{hellgelb}{rgb}{1,1,0.9}
\definecolor{colKeys}{rgb}{0,0,1}
\definecolor{colIdentifier}{rgb}{0,0,0}
\definecolor{colComments}{rgb}{1,0,0}
\definecolor{colString}{rgb}{0,0.5,0}
\definecolor{hellgrau}{rgb}{0.96,0.96,0.96}
\lstset{
    float=hbp,
    basicstyle=\ttfamily\color{black}\small\smaller,
    identifierstyle=\color{black},
    keywordstyle=\color{black},
    stringstyle=\color{black},
    commentstyle=\color{gray},
    columns=flexible,
    tabsize=2,
    frame=single,
    extendedchars=true,
    showspaces=false,
    showstringspaces=false,
    numbers=left,
    numberstyle=\tiny,
    breaklines=true,
    backgroundcolor=\color{hellgrau},
    breakautoindent=true
}

%%% END Article customizations


%%% The "real" document content comes below...

%\title{Ausgewählte Themen der Informatik in Theorie und Praxis}
%\author{Jens Thielemann}
%\date{} % Activate to display a given date or no date (if empty),
         % otherwise the current date is printed 

\begin{document}
%\maketitle
\tableofcontents 

\newpage

% % !TEX TS-program = pdflatex
% !TEX encoding = UTF-8 Unicode

% This is a simple template for a LaTeX document using the "article" class.
% See "book", "report", "letter" for other types of document.

% \documentclass[11pt]{article} % use larger type; default would be 10pt
\documentclass[
    10pt, % Schriftgröße
    DIV12,
    english, % für Umlaute, Silbentrennung etc.
    a5paper, % Papierformat
    twoside, % zweiseitiges Dokument
    titlepage, % es wird eine Titelseite verwendet
    parskip=half, % Abstand zwischen Absätzen (halbe Zeile)
    headings=small, % Größe der Überschriften verkleinern
    listof=totoc, % Verzeichnisse im Inhaltsverzeichnis aufführen
    bibliography=totoc, % Literaturverzeichnis im Inhaltsverzeichnis aufführen
    index=totoc, % Index im Inhaltsverzeichnis aufführen
    captions=tableheading, % Beschriftung von Tabellen unterhalb ausgeben
    final % Status des Dokuments (final/draft)
]{scrbook}

\usepackage[utf8]{inputenc} % set input encoding (not needed with XeLaTeX)
\usepackage[T1]{fontenc}
\usepackage{textcomp} % Euro-Zeichen etc.

\usepackage[
	automark, % Kapitelangaben in Kopfzeile automatisch erstellen
	headsepline, % Trennlinie unter Kopfzeile
	ilines % Trennlinie linksbündig ausrichten
]{scrlayer-scrpage}

% Anpassung an Landessprache ---------------------------------------------------
\usepackage[english]{babel}

%%% Examples of Article customizations
% These packages are optional, depending whether you want the features they provide.
% See the LaTeX Companion or other references for full information.

\usepackage{lmodern} % bessere Fonts
\usepackage{relsize} % Schriftgröße relativ festlegen

%%% PAGE DIMENSIONS
% \usepackage{geometry} % to change the page dimensions
\usepackage{setspace}
\usepackage[paperwidth=17cm,paperheight=24cm]{geometry}
% \geometry{a4paper} % or letterpaper (US) or a5paper or....
% \geometry{margin=2in} % for example, change the margins to 2 inches all round
% \geometry{landscape} % set up the page for landscape
%   read geometry.pdf for detailed page layout information
\geometry{left=20mm,right=20mm,top=15mm,bottom=20mm}

\usepackage{graphicx} % support the \includegraphics command and options

% \usepackage[parfill]{parskip} % Activate to begin paragraphs with an empty line rather than an indent

%%% PACKAGES
\usepackage{trfsigns}
\usepackage{booktabs} % for much better looking tables
\usepackage{array} % for better arrays (eg matrices) in maths
\usepackage{paralist} % very flexible & customisable lists (eg. enumerate/itemize, etc.)
\usepackage{verbatim} % adds environment for commenting out blocks of text & for better verbatim
\usepackage{subfig} % make it possible to include more than one captioned figure/table in a single float
\usepackage{float}
\usepackage{amsmath} %
\usepackage{amssymb}
\usepackage{tikz} % Draw sig­nal flow graphs
\usetikzlibrary{dsp,chains}
% These packages are all incorporated in the memoir class to one degree or another...

%%% HEADERS & FOOTERS
% \usepackage{fancyhdr} % This should be set AFTER setting up the page geometry
% \pagestyle{fancy} % options: empty , plain , fancy
% \renewcommand{\headrulewidth}{0pt} % customise the layout...
% \lhead{}\chead{}\rhead{}
% \lfoot{}\cfoot{\thepage}\rfoot{}

% Kopf- und Fußzeilen ----------------------------------------------------------
\pagestyle{scrheadings}
% Kopf- und Fußzeile auch auf Kapitelanfangsseiten
\renewcommand*{\chapterpagestyle}{scrheadings} 
% Schriftform der Kopfzeile
\renewcommand{\headfont}{\normalfont}

% Kopfzeile
\ihead{\textit{\headmark}}
\chead{}
\ohead{\pagemark}
\setlength{\headheight}{10mm} % Höhe der Kopfzeile
\setheadsepline[text]{0.4pt} % Trennlinie unter Kopfzeile

% Fußzeile
\ifoot{}
\cfoot{}
\ofoot{}

\setlength{\footskip}{20pt}

%%% SECTION TITLE APPEARANCE
%\usepackage{sectsty}
%\allsectionsfont{\sffamily\mdseries\upshape} % (See the fntguide.pdf for font help)
% (This matches ConTeXt defaults)

\RedeclareSectionCommand[
  beforeskip=-1sp,
  afterskip=2\baselineskip]{chapter}
\RedeclareSectionCommand[
  beforeskip=-\baselineskip,
  afterskip=.5\baselineskip]{section}
\RedeclareSectionCommand[
  beforeskip=-.75\baselineskip,
  afterskip=.5\baselineskip]{subsection}
\RedeclareSectionCommand[
  beforeskip=-.5\baselineskip,
  afterskip=.25\baselineskip]{subsubsection}
\RedeclareSectionCommand[
  beforeskip=.5\baselineskip,
  afterskip=-1em]{paragraph}
\RedeclareSectionCommand[
  beforeskip=-.5\baselineskip,
  afterskip=-1em]{subparagraph}

%%% ToC (table of contents) APPEARANCE
\usepackage[nottoc,notlof,notlot]{tocbibind} % Put the bibliography in the ToC
\usepackage[titles,subfigure]{tocloft} % Alter the style of the Table of Contents
\renewcommand{\cftsecfont}{\rmfamily\mdseries\upshape}
\renewcommand{\cftsecpagefont}{\rmfamily\mdseries\upshape} % No bold!

\DeclareMathAlphabet{\mathpzc}{OT1}{pzc}{m}{it}
\newcommand{\z}{\mathpzc{z}}

\usepackage{xcolor}
\definecolor{instruction}{gray}{0.80}

\newtheorem{beispiel}{Beispiel}[chapter]

% zum Einbinden von Programmcode -----------------------------------------------
\usepackage{listings}
\usepackage{xcolor} 
\definecolor{hellgelb}{rgb}{1,1,0.9}
\definecolor{colKeys}{rgb}{0,0,1}
\definecolor{colIdentifier}{rgb}{0,0,0}
\definecolor{colComments}{rgb}{1,0,0}
\definecolor{colString}{rgb}{0,0.5,0}
\definecolor{hellgrau}{rgb}{0.96,0.96,0.96}
\lstset{
    float=hbp,
    basicstyle=\ttfamily\color{black}\small\smaller,
    identifierstyle=\color{black},
    keywordstyle=\color{black},
    stringstyle=\color{black},
    commentstyle=\color{gray},
    columns=flexible,
    tabsize=2,
    frame=single,
    extendedchars=true,
    showspaces=false,
    showstringspaces=false,
    numbers=left,
    numberstyle=\tiny,
    breaklines=true,
    backgroundcolor=\color{hellgrau},
    breakautoindent=true
}

%%% END Article customizations


%%% The "real" document content comes below...

%\title{Ausgewählte Themen der Informatik in Theorie und Praxis}
%\author{Jens Thielemann}
%\date{} % Activate to display a given date or no date (if empty),
         % otherwise the current date is printed 

\begin{document}
%\maketitle
\tableofcontents 

\newpage

% % !TEX TS-program = pdflatex
% !TEX encoding = UTF-8 Unicode

% This is a simple template for a LaTeX document using the "article" class.
% See "book", "report", "letter" for other types of document.

% \documentclass[11pt]{article} % use larger type; default would be 10pt
\documentclass[
    10pt, % Schriftgröße
    DIV12,
    english, % für Umlaute, Silbentrennung etc.
    a5paper, % Papierformat
    twoside, % zweiseitiges Dokument
    titlepage, % es wird eine Titelseite verwendet
    parskip=half, % Abstand zwischen Absätzen (halbe Zeile)
    headings=small, % Größe der Überschriften verkleinern
    listof=totoc, % Verzeichnisse im Inhaltsverzeichnis aufführen
    bibliography=totoc, % Literaturverzeichnis im Inhaltsverzeichnis aufführen
    index=totoc, % Index im Inhaltsverzeichnis aufführen
    captions=tableheading, % Beschriftung von Tabellen unterhalb ausgeben
    final % Status des Dokuments (final/draft)
]{scrbook}

\usepackage[utf8]{inputenc} % set input encoding (not needed with XeLaTeX)
\usepackage[T1]{fontenc}
\usepackage{textcomp} % Euro-Zeichen etc.

\usepackage[
	automark, % Kapitelangaben in Kopfzeile automatisch erstellen
	headsepline, % Trennlinie unter Kopfzeile
	ilines % Trennlinie linksbündig ausrichten
]{scrlayer-scrpage}

% Anpassung an Landessprache ---------------------------------------------------
\usepackage[english]{babel}

%%% Examples of Article customizations
% These packages are optional, depending whether you want the features they provide.
% See the LaTeX Companion or other references for full information.

\usepackage{lmodern} % bessere Fonts
\usepackage{relsize} % Schriftgröße relativ festlegen

%%% PAGE DIMENSIONS
% \usepackage{geometry} % to change the page dimensions
\usepackage{setspace}
\usepackage[paperwidth=17cm,paperheight=24cm]{geometry}
% \geometry{a4paper} % or letterpaper (US) or a5paper or....
% \geometry{margin=2in} % for example, change the margins to 2 inches all round
% \geometry{landscape} % set up the page for landscape
%   read geometry.pdf for detailed page layout information
\geometry{left=20mm,right=20mm,top=15mm,bottom=20mm}

\usepackage{graphicx} % support the \includegraphics command and options

% \usepackage[parfill]{parskip} % Activate to begin paragraphs with an empty line rather than an indent

%%% PACKAGES
\usepackage{trfsigns}
\usepackage{booktabs} % for much better looking tables
\usepackage{array} % for better arrays (eg matrices) in maths
\usepackage{paralist} % very flexible & customisable lists (eg. enumerate/itemize, etc.)
\usepackage{verbatim} % adds environment for commenting out blocks of text & for better verbatim
\usepackage{subfig} % make it possible to include more than one captioned figure/table in a single float
\usepackage{float}
\usepackage{amsmath} %
\usepackage{amssymb}
\usepackage{tikz} % Draw sig­nal flow graphs
\usetikzlibrary{dsp,chains}
% These packages are all incorporated in the memoir class to one degree or another...

%%% HEADERS & FOOTERS
% \usepackage{fancyhdr} % This should be set AFTER setting up the page geometry
% \pagestyle{fancy} % options: empty , plain , fancy
% \renewcommand{\headrulewidth}{0pt} % customise the layout...
% \lhead{}\chead{}\rhead{}
% \lfoot{}\cfoot{\thepage}\rfoot{}

% Kopf- und Fußzeilen ----------------------------------------------------------
\pagestyle{scrheadings}
% Kopf- und Fußzeile auch auf Kapitelanfangsseiten
\renewcommand*{\chapterpagestyle}{scrheadings} 
% Schriftform der Kopfzeile
\renewcommand{\headfont}{\normalfont}

% Kopfzeile
\ihead{\textit{\headmark}}
\chead{}
\ohead{\pagemark}
\setlength{\headheight}{10mm} % Höhe der Kopfzeile
\setheadsepline[text]{0.4pt} % Trennlinie unter Kopfzeile

% Fußzeile
\ifoot{}
\cfoot{}
\ofoot{}

\setlength{\footskip}{20pt}

%%% SECTION TITLE APPEARANCE
%\usepackage{sectsty}
%\allsectionsfont{\sffamily\mdseries\upshape} % (See the fntguide.pdf for font help)
% (This matches ConTeXt defaults)

\RedeclareSectionCommand[
  beforeskip=-1sp,
  afterskip=2\baselineskip]{chapter}
\RedeclareSectionCommand[
  beforeskip=-\baselineskip,
  afterskip=.5\baselineskip]{section}
\RedeclareSectionCommand[
  beforeskip=-.75\baselineskip,
  afterskip=.5\baselineskip]{subsection}
\RedeclareSectionCommand[
  beforeskip=-.5\baselineskip,
  afterskip=.25\baselineskip]{subsubsection}
\RedeclareSectionCommand[
  beforeskip=.5\baselineskip,
  afterskip=-1em]{paragraph}
\RedeclareSectionCommand[
  beforeskip=-.5\baselineskip,
  afterskip=-1em]{subparagraph}

%%% ToC (table of contents) APPEARANCE
\usepackage[nottoc,notlof,notlot]{tocbibind} % Put the bibliography in the ToC
\usepackage[titles,subfigure]{tocloft} % Alter the style of the Table of Contents
\renewcommand{\cftsecfont}{\rmfamily\mdseries\upshape}
\renewcommand{\cftsecpagefont}{\rmfamily\mdseries\upshape} % No bold!

\DeclareMathAlphabet{\mathpzc}{OT1}{pzc}{m}{it}
\newcommand{\z}{\mathpzc{z}}

\usepackage{xcolor}
\definecolor{instruction}{gray}{0.80}

\newtheorem{beispiel}{Beispiel}[chapter]

% zum Einbinden von Programmcode -----------------------------------------------
\usepackage{listings}
\usepackage{xcolor} 
\definecolor{hellgelb}{rgb}{1,1,0.9}
\definecolor{colKeys}{rgb}{0,0,1}
\definecolor{colIdentifier}{rgb}{0,0,0}
\definecolor{colComments}{rgb}{1,0,0}
\definecolor{colString}{rgb}{0,0.5,0}
\definecolor{hellgrau}{rgb}{0.96,0.96,0.96}
\lstset{
    float=hbp,
    basicstyle=\ttfamily\color{black}\small\smaller,
    identifierstyle=\color{black},
    keywordstyle=\color{black},
    stringstyle=\color{black},
    commentstyle=\color{gray},
    columns=flexible,
    tabsize=2,
    frame=single,
    extendedchars=true,
    showspaces=false,
    showstringspaces=false,
    numbers=left,
    numberstyle=\tiny,
    breaklines=true,
    backgroundcolor=\color{hellgrau},
    breakautoindent=true
}

%%% END Article customizations


%%% The "real" document content comes below...

%\title{Ausgewählte Themen der Informatik in Theorie und Praxis}
%\author{Jens Thielemann}
%\date{} % Activate to display a given date or no date (if empty),
         % otherwise the current date is printed 

\begin{document}
%\maketitle
\tableofcontents 

\newpage

% % !TEX TS-program = pdflatex
% !TEX encoding = UTF-8 Unicode

% This is a simple template for a LaTeX document using the "article" class.
% See "book", "report", "letter" for other types of document.

% \documentclass[11pt]{article} % use larger type; default would be 10pt
\documentclass[
    10pt, % Schriftgröße
    DIV12,
    english, % für Umlaute, Silbentrennung etc.
    a5paper, % Papierformat
    twoside, % zweiseitiges Dokument
    titlepage, % es wird eine Titelseite verwendet
    parskip=half, % Abstand zwischen Absätzen (halbe Zeile)
    headings=small, % Größe der Überschriften verkleinern
    listof=totoc, % Verzeichnisse im Inhaltsverzeichnis aufführen
    bibliography=totoc, % Literaturverzeichnis im Inhaltsverzeichnis aufführen
    index=totoc, % Index im Inhaltsverzeichnis aufführen
    captions=tableheading, % Beschriftung von Tabellen unterhalb ausgeben
    final % Status des Dokuments (final/draft)
]{scrbook}

\usepackage[utf8]{inputenc} % set input encoding (not needed with XeLaTeX)
\usepackage[T1]{fontenc}
\usepackage{textcomp} % Euro-Zeichen etc.

\usepackage[
	automark, % Kapitelangaben in Kopfzeile automatisch erstellen
	headsepline, % Trennlinie unter Kopfzeile
	ilines % Trennlinie linksbündig ausrichten
]{scrlayer-scrpage}

% Anpassung an Landessprache ---------------------------------------------------
\usepackage[english]{babel}

%%% Examples of Article customizations
% These packages are optional, depending whether you want the features they provide.
% See the LaTeX Companion or other references for full information.

\usepackage{lmodern} % bessere Fonts
\usepackage{relsize} % Schriftgröße relativ festlegen

%%% PAGE DIMENSIONS
% \usepackage{geometry} % to change the page dimensions
\usepackage{setspace}
\usepackage[paperwidth=17cm,paperheight=24cm]{geometry}
% \geometry{a4paper} % or letterpaper (US) or a5paper or....
% \geometry{margin=2in} % for example, change the margins to 2 inches all round
% \geometry{landscape} % set up the page for landscape
%   read geometry.pdf for detailed page layout information
\geometry{left=20mm,right=20mm,top=15mm,bottom=20mm}

\usepackage{graphicx} % support the \includegraphics command and options

% \usepackage[parfill]{parskip} % Activate to begin paragraphs with an empty line rather than an indent

%%% PACKAGES
\usepackage{trfsigns}
\usepackage{booktabs} % for much better looking tables
\usepackage{array} % for better arrays (eg matrices) in maths
\usepackage{paralist} % very flexible & customisable lists (eg. enumerate/itemize, etc.)
\usepackage{verbatim} % adds environment for commenting out blocks of text & for better verbatim
\usepackage{subfig} % make it possible to include more than one captioned figure/table in a single float
\usepackage{float}
\usepackage{amsmath} %
\usepackage{amssymb}
\usepackage{tikz} % Draw sig­nal flow graphs
\usetikzlibrary{dsp,chains}
% These packages are all incorporated in the memoir class to one degree or another...

%%% HEADERS & FOOTERS
% \usepackage{fancyhdr} % This should be set AFTER setting up the page geometry
% \pagestyle{fancy} % options: empty , plain , fancy
% \renewcommand{\headrulewidth}{0pt} % customise the layout...
% \lhead{}\chead{}\rhead{}
% \lfoot{}\cfoot{\thepage}\rfoot{}

% Kopf- und Fußzeilen ----------------------------------------------------------
\pagestyle{scrheadings}
% Kopf- und Fußzeile auch auf Kapitelanfangsseiten
\renewcommand*{\chapterpagestyle}{scrheadings} 
% Schriftform der Kopfzeile
\renewcommand{\headfont}{\normalfont}

% Kopfzeile
\ihead{\textit{\headmark}}
\chead{}
\ohead{\pagemark}
\setlength{\headheight}{10mm} % Höhe der Kopfzeile
\setheadsepline[text]{0.4pt} % Trennlinie unter Kopfzeile

% Fußzeile
\ifoot{}
\cfoot{}
\ofoot{}

\setlength{\footskip}{20pt}

%%% SECTION TITLE APPEARANCE
%\usepackage{sectsty}
%\allsectionsfont{\sffamily\mdseries\upshape} % (See the fntguide.pdf for font help)
% (This matches ConTeXt defaults)

\RedeclareSectionCommand[
  beforeskip=-1sp,
  afterskip=2\baselineskip]{chapter}
\RedeclareSectionCommand[
  beforeskip=-\baselineskip,
  afterskip=.5\baselineskip]{section}
\RedeclareSectionCommand[
  beforeskip=-.75\baselineskip,
  afterskip=.5\baselineskip]{subsection}
\RedeclareSectionCommand[
  beforeskip=-.5\baselineskip,
  afterskip=.25\baselineskip]{subsubsection}
\RedeclareSectionCommand[
  beforeskip=.5\baselineskip,
  afterskip=-1em]{paragraph}
\RedeclareSectionCommand[
  beforeskip=-.5\baselineskip,
  afterskip=-1em]{subparagraph}

%%% ToC (table of contents) APPEARANCE
\usepackage[nottoc,notlof,notlot]{tocbibind} % Put the bibliography in the ToC
\usepackage[titles,subfigure]{tocloft} % Alter the style of the Table of Contents
\renewcommand{\cftsecfont}{\rmfamily\mdseries\upshape}
\renewcommand{\cftsecpagefont}{\rmfamily\mdseries\upshape} % No bold!

\DeclareMathAlphabet{\mathpzc}{OT1}{pzc}{m}{it}
\newcommand{\z}{\mathpzc{z}}

\usepackage{xcolor}
\definecolor{instruction}{gray}{0.80}

\newtheorem{beispiel}{Beispiel}[chapter]

% zum Einbinden von Programmcode -----------------------------------------------
\usepackage{listings}
\usepackage{xcolor} 
\definecolor{hellgelb}{rgb}{1,1,0.9}
\definecolor{colKeys}{rgb}{0,0,1}
\definecolor{colIdentifier}{rgb}{0,0,0}
\definecolor{colComments}{rgb}{1,0,0}
\definecolor{colString}{rgb}{0,0.5,0}
\definecolor{hellgrau}{rgb}{0.96,0.96,0.96}
\lstset{
    float=hbp,
    basicstyle=\ttfamily\color{black}\small\smaller,
    identifierstyle=\color{black},
    keywordstyle=\color{black},
    stringstyle=\color{black},
    commentstyle=\color{gray},
    columns=flexible,
    tabsize=2,
    frame=single,
    extendedchars=true,
    showspaces=false,
    showstringspaces=false,
    numbers=left,
    numberstyle=\tiny,
    breaklines=true,
    backgroundcolor=\color{hellgrau},
    breakautoindent=true
}

%%% END Article customizations


%%% The "real" document content comes below...

%\title{Ausgewählte Themen der Informatik in Theorie und Praxis}
%\author{Jens Thielemann}
%\date{} % Activate to display a given date or no date (if empty),
         % otherwise the current date is printed 

\begin{document}
%\maketitle
\tableofcontents 

\newpage

% \include{mean_filter}
\chapter{Mean Filter}

\section{Moving Average}

\begin{equation}
y_{k}=\frac{1}{N}\sum_{n=0}^{N-1}x_{k-n}\nonumber
\end{equation}
\begin{equation}
y_{k}=\frac{1}{N}x_k+y_{k-1}-\frac{1}{N}x_{k-N}\nonumber
\end{equation}
\begin{equation}
y_{k}=y_{k-1}+\frac{1}{N}\left(x_{k}-x_{k-N}\right)\nonumber
\end{equation}
\\
\\
\begin{equation}
y_{k}=\frac{1}{N}\sum_{n=0}^{N-1}x_{k-n}\nonumber
\end{equation}
\begin{equation}
y_{k}=\frac{1}{N}x_k+y_{k-1}-\frac{1}{N}x_{k-N}\nonumber
\end{equation}
\begin{equation}
y_{k}=\frac{1}{N}x_k+y_{k-1}\frac{N-1}{N}\nonumber
\end{equation}
\begin{equation}
y_{k}=y_{k-1}+\frac{1}{N}\left(x_{k}-y_{k-1}\right)\nonumber
\end{equation}

\section{PT1-Filter}

\begin{equation}
Y_{(s)} =  \frac{K}{1+T s} \: X_{(s)}\nonumber
\end{equation}
\begin{equation}
Y_{(s)}+ T \: s\: Y_{(s)} = K \:X_{(s)} \:\: \laplace \:\: y_{(t)}+T\:\dot{y}_{(t)}=K\: x_{(t)}\nonumber
\end{equation}
\begin{equation}
y_{k}+T\:\frac{y_k-y_{k-1}}{dt}=K\: x_k\nonumber
\end{equation}
\begin{equation}
y_{k}+\frac{dt}{T}\:y_k=y_{k-1}+K\:\frac{dt}{T}\:x_k\nonumber
\end{equation}
\begin{equation}
y_{k}=\frac{T}{T+dt}\:y_{k-1}+K\:\frac{dt}{T+dt}\:x_k\nonumber
\end{equation}
\\
\begin{equation}
\frac{T}{T+dt}\:y_{k-1}=y_{k-1}-\frac{dt}{T+dt}\:y_{k-1}\nonumber
\end{equation}
\\
\begin{equation}
y_{k}=y_{k-1}+\frac{dt}{T+dt}\:\left(K\:x_k-y_{k-1}\right)\nonumber
\end{equation}
\\
\\
\begin{equation}
y_{k}=y_{k-1}+\frac{dt}{T+dt}\:\left(K\:x_k-y_{k-1}\right)\nonumber
\end{equation}
\\
\begin{equation}
K=1\nonumber
\end{equation}
\begin{equation}
\frac{dt}{T+dt} = \frac{1}{N}\nonumber
\end{equation}
\\
\begin{equation}
y_{k}=y_{k-1}+\frac{1}{N}\left(x_{k}-y_{k-1}\right)\nonumber
\end{equation}
\end{document}





\chapter{Mean Filter}

\section{Moving Average}

\begin{equation}
y_{k}=\frac{1}{N}\sum_{n=0}^{N-1}x_{k-n}\nonumber
\end{equation}
\begin{equation}
y_{k}=\frac{1}{N}x_k+y_{k-1}-\frac{1}{N}x_{k-N}\nonumber
\end{equation}
\begin{equation}
y_{k}=y_{k-1}+\frac{1}{N}\left(x_{k}-x_{k-N}\right)\nonumber
\end{equation}
\\
\\
\begin{equation}
y_{k}=\frac{1}{N}\sum_{n=0}^{N-1}x_{k-n}\nonumber
\end{equation}
\begin{equation}
y_{k}=\frac{1}{N}x_k+y_{k-1}-\frac{1}{N}x_{k-N}\nonumber
\end{equation}
\begin{equation}
y_{k}=\frac{1}{N}x_k+y_{k-1}\frac{N-1}{N}\nonumber
\end{equation}
\begin{equation}
y_{k}=y_{k-1}+\frac{1}{N}\left(x_{k}-y_{k-1}\right)\nonumber
\end{equation}

\section{PT1-Filter}

\begin{equation}
Y_{(s)} =  \frac{K}{1+T s} \: X_{(s)}\nonumber
\end{equation}
\begin{equation}
Y_{(s)}+ T \: s\: Y_{(s)} = K \:X_{(s)} \:\: \laplace \:\: y_{(t)}+T\:\dot{y}_{(t)}=K\: x_{(t)}\nonumber
\end{equation}
\begin{equation}
y_{k}+T\:\frac{y_k-y_{k-1}}{dt}=K\: x_k\nonumber
\end{equation}
\begin{equation}
y_{k}+\frac{dt}{T}\:y_k=y_{k-1}+K\:\frac{dt}{T}\:x_k\nonumber
\end{equation}
\begin{equation}
y_{k}=\frac{T}{T+dt}\:y_{k-1}+K\:\frac{dt}{T+dt}\:x_k\nonumber
\end{equation}
\\
\begin{equation}
\frac{T}{T+dt}\:y_{k-1}=y_{k-1}-\frac{dt}{T+dt}\:y_{k-1}\nonumber
\end{equation}
\\
\begin{equation}
y_{k}=y_{k-1}+\frac{dt}{T+dt}\:\left(K\:x_k-y_{k-1}\right)\nonumber
\end{equation}
\\
\\
\begin{equation}
y_{k}=y_{k-1}+\frac{dt}{T+dt}\:\left(K\:x_k-y_{k-1}\right)\nonumber
\end{equation}
\\
\begin{equation}
K=1\nonumber
\end{equation}
\begin{equation}
\frac{dt}{T+dt} = \frac{1}{N}\nonumber
\end{equation}
\\
\begin{equation}
y_{k}=y_{k-1}+\frac{1}{N}\left(x_{k}-y_{k-1}\right)\nonumber
\end{equation}
\end{document}





\chapter{Mean Filter}

\section{Moving Average}

\begin{equation}
y_{k}=\frac{1}{N}\sum_{n=0}^{N-1}x_{k-n}\nonumber
\end{equation}
\begin{equation}
y_{k}=\frac{1}{N}x_k+y_{k-1}-\frac{1}{N}x_{k-N}\nonumber
\end{equation}
\begin{equation}
y_{k}=y_{k-1}+\frac{1}{N}\left(x_{k}-x_{k-N}\right)\nonumber
\end{equation}
\\
\\
\begin{equation}
y_{k}=\frac{1}{N}\sum_{n=0}^{N-1}x_{k-n}\nonumber
\end{equation}
\begin{equation}
y_{k}=\frac{1}{N}x_k+y_{k-1}-\frac{1}{N}x_{k-N}\nonumber
\end{equation}
\begin{equation}
y_{k}=\frac{1}{N}x_k+y_{k-1}\frac{N-1}{N}\nonumber
\end{equation}
\begin{equation}
y_{k}=y_{k-1}+\frac{1}{N}\left(x_{k}-y_{k-1}\right)\nonumber
\end{equation}

\section{PT1-Filter}

\begin{equation}
Y_{(s)} =  \frac{K}{1+T s} \: X_{(s)}\nonumber
\end{equation}
\begin{equation}
Y_{(s)}+ T \: s\: Y_{(s)} = K \:X_{(s)} \:\: \laplace \:\: y_{(t)}+T\:\dot{y}_{(t)}=K\: x_{(t)}\nonumber
\end{equation}
\begin{equation}
y_{k}+T\:\frac{y_k-y_{k-1}}{dt}=K\: x_k\nonumber
\end{equation}
\begin{equation}
y_{k}+\frac{dt}{T}\:y_k=y_{k-1}+K\:\frac{dt}{T}\:x_k\nonumber
\end{equation}
\begin{equation}
y_{k}=\frac{T}{T+dt}\:y_{k-1}+K\:\frac{dt}{T+dt}\:x_k\nonumber
\end{equation}
\\
\begin{equation}
\frac{T}{T+dt}\:y_{k-1}=y_{k-1}-\frac{dt}{T+dt}\:y_{k-1}\nonumber
\end{equation}
\\
\begin{equation}
y_{k}=y_{k-1}+\frac{dt}{T+dt}\:\left(K\:x_k-y_{k-1}\right)\nonumber
\end{equation}
\\
\\
\begin{equation}
y_{k}=y_{k-1}+\frac{dt}{T+dt}\:\left(K\:x_k-y_{k-1}\right)\nonumber
\end{equation}
\\
\begin{equation}
K=1\nonumber
\end{equation}
\begin{equation}
\frac{dt}{T+dt} = \frac{1}{N}\nonumber
\end{equation}
\\
\begin{equation}
y_{k}=y_{k-1}+\frac{1}{N}\left(x_{k}-y_{k-1}\right)\nonumber
\end{equation}
\end{document}





\chapter{Mean Filter}

\section{Moving Average}

\begin{equation}
y_{k}=\frac{1}{N}\sum_{n=0}^{N-1}x_{k-n}\nonumber
\end{equation}
\begin{equation}
y_{k}=\frac{1}{N}x_k+y_{k-1}-\frac{1}{N}x_{k-N}\nonumber
\end{equation}
\begin{equation}
y_{k}=y_{k-1}+\frac{1}{N}\left(x_{k}-x_{k-N}\right)\nonumber
\end{equation}
\\
\\
\begin{equation}
y_{k}=\frac{1}{N}\sum_{n=0}^{N-1}x_{k-n}\nonumber
\end{equation}
\begin{equation}
y_{k}=\frac{1}{N}x_k+y_{k-1}-\frac{1}{N}x_{k-N}\nonumber
\end{equation}
\begin{equation}
y_{k}=\frac{1}{N}x_k+y_{k-1}\frac{N-1}{N}\nonumber
\end{equation}
\begin{equation}
y_{k}=y_{k-1}+\frac{1}{N}\left(x_{k}-y_{k-1}\right)\nonumber
\end{equation}

\section{PT1-Filter}

\begin{equation}
Y_{(s)} =  \frac{K}{1+T s} \: X_{(s)}\nonumber
\end{equation}
\begin{equation}
Y_{(s)}+ T \: s\: Y_{(s)} = K \:X_{(s)} \:\: \laplace \:\: y_{(t)}+T\:\dot{y}_{(t)}=K\: x_{(t)}\nonumber
\end{equation}
\begin{equation}
y_{k}+T\:\frac{y_k-y_{k-1}}{dt}=K\: x_k\nonumber
\end{equation}
\begin{equation}
y_{k}+\frac{dt}{T}\:y_k=y_{k-1}+K\:\frac{dt}{T}\:x_k\nonumber
\end{equation}
\begin{equation}
y_{k}=\frac{T}{T+dt}\:y_{k-1}+K\:\frac{dt}{T+dt}\:x_k\nonumber
\end{equation}
\\
\begin{equation}
\frac{T}{T+dt}\:y_{k-1}=y_{k-1}-\frac{dt}{T+dt}\:y_{k-1}\nonumber
\end{equation}
\\
\begin{equation}
y_{k}=y_{k-1}+\frac{dt}{T+dt}\:\left(K\:x_k-y_{k-1}\right)\nonumber
\end{equation}
\\
\\
\begin{equation}
y_{k}=y_{k-1}+\frac{dt}{T+dt}\:\left(K\:x_k-y_{k-1}\right)\nonumber
\end{equation}
\\
\begin{equation}
K=1\nonumber
\end{equation}
\begin{equation}
\frac{dt}{T+dt} = \frac{1}{N}\nonumber
\end{equation}
\\
\begin{equation}
y_{k}=y_{k-1}+\frac{1}{N}\left(x_{k}-y_{k-1}\right)\nonumber
\end{equation}


\section{Low Pass Filterr}

\begin{equation}
H_{(s)} = \frac{2\pi f_c}{2\pi f_c+s}=\frac{1}{1+\frac{1}{2\pi f_c}s}= \frac{K}{1+T\:s} \nonumber
\end{equation}

\begin{equation}
H_{(z)}=Z\big\{H_{(s)}  H_{0(s)} \big\} \nonumber
\end{equation}

\begin{equation}
H_{(z)}=Z\bigg\{H_{(s)}  \frac{1}{s (1-e^{-s\:T_a})}\bigg\}=\frac{z-1}{z} Z\bigg\{\frac{H_{(s)}}{s}\bigg\} \nonumber
\end{equation}

\begin{equation}
Z\bigg\{\frac{1}{s}\bigg\}=\frac{z}{z-1}
Z\bigg\{\frac{a}{a+s}\bigg\}=\frac{1-e^{-a\:T_a}}{z-e^{-a\: T_a}} \nonumber
\end{equation}

\begin{equation}
H_{(z)}=\frac{z-1}{z} \frac{z}{z-1} K \frac{1-e^{-\frac{T_a}{T}}}{z-e^{-\frac{T_a}{T}}}=K \frac{1-e^{-\frac{T_a}{T}}}{z-e^{-\frac{T_a}{T}}}= \frac{K (1-e^{-\frac{T_a}{T}}) z^{-1}}{(1-e^{-\frac{T_a}{T}}) z^{-1}} \nonumber
\end{equation}

\begin{equation}
H_{(z)}=\frac{b_0+b_1\:z^{-1}}{a_0+a_1\:z^{-1}} \:\: \laplace\:\: y_k=b_1\:u_{k-1}-a_1\:y_{k-1}\approx b_1\:u_{k}-a_1\:y_{k-1} \nonumber
\end{equation}

\begin{equation}
b_0=0, b_1=K(1-e^{-\frac{T_a}{T}}), a_0=1 (\rightarrow normed), a_1=-e^{-\frac{T_a}{T}} \nonumber
\end{equation}


\end{document}
